\begin{circuslaw}{distr-prefix-par-4}
\begin{circusaction*}
  c~!~x \then \Skip \circseq (A_1 \lpar ns_1 | cs | ns_2 \rpar A_2)
%  \also
  ~ \equiv ~
%  \also
  (c~!~x \then \Skip \circseq A_1) \lpar ns_1 | cs | ns_2 \rpar (c~?~\anyvar \then A_2)
  \also
  \provided ~ c \in cs \, \provand \, \mbox{$y$ is not free in $A_2$}
\end{circusaction*}
\end{circuslaw}
