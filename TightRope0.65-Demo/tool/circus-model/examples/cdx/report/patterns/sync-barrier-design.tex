\begin{pattern}{sync-barrier-design}
\begin{circusaction*}
  \circblockopen
    (\circmu X \circspot start \then A_1 \circseq done \then \Skip \circseq sync \then X)
    \also
    \t1 \lpar ns_1 | cs_1 | ns_2 \cup \dots \cup ns_n \rpar
    \also
    (\circmu X \circspot start \then A_2 \circseq done \then \Skip \circseq sync \then X)
    \also
    \t1 \lpar ns_2 | cs_2 | ns_3 \cup \dots \cup ns_n \rpar
    \also
    \t2 \dots
    \also
    \t1 \lpar ns_{n-1} | cs_{n-1} | ns_{n} \rpar
    \also
    (\circmu X \circspot start \then A_n \circseq done \then \Skip \circseq sync \then X)
  \circblockclose
  \also
  \t1 \refby
  \also
  \circblockopen
    \circblockopen
      \highlight{\circmu X \circspot reset \then start \then \Skip \circseq sync \then X}
    \circblockclose
    \also
    \t1 \lpar \emptyset | \lchanset reset, start, sync \rchanset | \emptyset \rpar
    \also
    \circblockopen
      (\circmu X \circspot start \then A_1 \circseq notify~!~1 \then \Skip \circseq sync \then X)
      \also
      \t1 \lpar ns_1 | cs_1 \setdiff \lchanset done \rchanset | ns_2 \cup \dots \cup ns_n \rpar
      \also
      (\circmu X \circspot start \then A_2 \circseq notify~!~2 \then \Skip \circseq sync \then X)
      \also
      \t1 \lpar ns_2 | cs_2 \setdiff \lchanset done \rchanset | ns_3 \cup \dots \cup ns_n \rpar
      \also
      \t2 \dots
      \also
      \t1 \lpar ns_{n-1} | cs_{n-1} \setdiff \lchanset done \rchanset | ns_{n} \rpar
      \also
      (\circmu X \circspot start \then A_n \circseq notify~!~n \then \Skip \circseq sync \then X)
    \circblockclose
    \also
    \t1 \lpar ns_1 \cup \dots \cup ns_n | \lchanset start, notify, sync \rchanset | \emptyset \rpar
    \also
    \circblockopen
      \circvar active : \power ~ (1 \upto n) \circspot
      \\
      \t1 \circmu X \circspot
      \circblockopen
        (reset \then active := 1 \upto n)
        \\
        \t1 \extchoice
        \\
        (notify~?~x \then
        \circblockopen
          active := active \setdiff \{x\} \circseq
          \\
          \circif active = \emptyset \circthen done \then \Skip\\
          \circelse ~ \lnot active = \emptyset \circthen \Skip\\
          \circfi
        \circblockclose)
      \circblockclose
      \circseq X
    \circblockclose
  \circblockclose
  \circhide \lchanset reset, notify \rchanset
  \also
  \provided \; \lchanset reset, start, done, sync \rchanset \subseteq cs_i \land \{reset, start, done, sync\} \cap \usedC(A_i) = \emptyset \; \mbox{for all} \; i : 1 \upto n
  \also
  \provand \; \mbox{$notify$ is a fresh channel of type $\nat$}
\end{circusaction*}
\end{pattern}
