\documentclass{article}
\usepackage{fullpage}
\usepackage{wasysym}
\usepackage{verbatim}
\usepackage[usenames,dvipsnames]{color}
\usepackage[colour]{circus}
\usepackage{hijac}

\newcommand{\reducevspaceaftersection}{\vspace{-1em}}

\title{{\Circus} Model for the SCJ Mission Framework}

\author{Frank Zeyda}

\begin{document}

\maketitle

\tableofcontents

\newpage

%%%%%%%%%%%%%%%%%%%%%%%%%%%%%%%%%%%%%%%%%%%%%%%%%%%%%%%%%%%%%%%%%%%%%%%%%%%%%%%%
%                                   Notation                                   %
%%%%%%%%%%%%%%%%%%%%%%%%%%%%%%%%%%%%%%%%%%%%%%%%%%%%%%%%%%%%%%%%%%%%%%%%%%%%%%%%

\section{Notation}

To highlight the channels that a process synchronises on we introduce a new clause $\circexposes$ as part of a process definition. An example of its use is given below.
%
\begin{circusbox}
\nounparsedcolour
\begin{circus*}
  \circprocess ~ MainSafelet ~ \circdef
  \\
  \t1 (SafeletFW \lpar MainSafeletAppChan \rpar MainSafeletApp) \circhide MainSafeletAppChan
\end{circus*}
%
\begin{circus*}
    \circexposes ~ \lchanset start\_sequencer, done\_sequencer \rchanset
\end{circus*}
\end{circusbox}
%
The $\circexposes$ clause indicates that the process synchronises on the two channels $start\_sequencer$ and $done\_sequencer$. The clause is solely for documentary purposes and does not have any semantic impact.


\newpage

%%%%%%%%%%%%%%%%%%%%%%%%%%%%%%%%%%%%%%%%%%%%%%%%%%%%%%%%%%%%%%%%%%%%%%%%%%%%%%%%
%                                   Prelude                                    %
%%%%%%%%%%%%%%%%%%%%%%%%%%%%%%%%%%%%%%%%%%%%%%%%%%%%%%%%%%%%%%%%%%%%%%%%%%%%%%%%

\section{Prelude}

Start of a new section $scj\_prelude$ with $circus\_toolkit$ as its parent.
%
\begin{zsection}
  \SECTION ~ scj\_prelude ~ \parents ~ circus\_toolkit
\end{zsection}

\subsection{Types}

%%Zword \bool BOOL
%%Zword \btrue TRUE
%%Zword \bfalse FALSE

The type $BOOL$ is used for boolean values in Z at the \emph{specification} level.
%
\begin{zed}
  BOOL ::= TRUE | FALSE
\end{zed}
%
We note that at the program level we use the type $boolean$ rather than $BOOL$.

\subsection{Functions}

The unary relation $distinct$ captures that the elements of a sequence are distinct.
%
\begin{circusflow}
\begin{zed}
  \relation ~ ~ (distinct ~ \_)
\end{zed}
%
\begin{gendef}[X]
  distinct ~ \_ : \power ~ (\iseq X)
\end{gendef}
\end{circusflow}
%
This function is used to specify uniqueness of the various identifiers introduced.

\subsection{References}

We do not actually define a semantics for references. Instead, we provide loose definitions of operators that allow us to some extent to type-check specifications that make use of reference types.

%%Zpreword \circreftype reftype
%%Zpreword \circref ref
%%Zpreword \circderef deref
%%Zinword \circrefassign {:=}

We first introduce a generic type $REF$ that acts as the semantic domain for references over a certain value type. This is simply done by identifying $REF$ with the value type itself.% cross some marker type $REFTAG$.
%
\begin{circusflow}
%\begin{zed}
%  [REFTAG]
%\end{zed}
%
%\begin{zed}
%  REF[X] ~ == ~ X \cross REFTAG
%\end{zed}
\begin{zed}
  REF[X] ~ == ~ X
\end{zed}
\end{circusflow}

\nid We next define a generic function that serves as the type constructor for references.
%
\begin{circusflow}
\begin{zed}
  \generic ~ ~ (\circreftype \_)
\end{zed}
%
\begin{zed}
  \circreftype X ~ == ~ REF[X]
\end{zed}
\end{circusflow}

\nid The generic referencing and dereferencing operators are loosely specified below.
%
\begin{gendef}[X]
  \circref : X \fun \circreftype X
  \also
  \circderef : \circreftype X \fun X
\end{gendef}
%
The fact that we identified $\circreftype X$ directly with $X$ is a design decision; a more strict approach may introduce a type different from $X$ here, such as $X \times REFTAG$ where $REFTAG$ is a new given type to `tag' value types as references. The pros of this tagging is that we get stricter type checking. The cons is that we cannot treat references to schemas nicely, namely when accessing components of the schema by selection. In order to maximise the parts of the model that can be checked by the tools, we opted for the more lenient design.

\subsection{Arrays}

%%Zpreword \circarray array

Arrays are modelled by sequences. For convenience, we introduce a type constructor for them.
%
\begin{zed}
  \circarray[X] ~ == ~ (\lambda T : \power X @ \circreftype~(\seq T))
\end{zed}
%
Note that the result is not the underlying sequence type but the corresponding reference type.

\subsection{{\OhCircus}}

%%Zpreword \circnew new

Below we provide support for the $\circnew$ keyword in {\OhCircus}.
%
\begin{gendef}[X]
  \circnew : \power X \fun \circreftype X
\end{gendef}
%
Again the specification is loose in that we do not define a semantics for the operator.

Special $\circnew$ operators of {\SCJCircus} are just equated with $\circnew$ above for type-checking purposes.

\begin{verbatim}
%%Zpreword \circnewI new
%%Zpreword \circnewM new
%%Zpreword \circnewR new
%%Zpreword \circnewP new
\end{verbatim}

%%Zword \circnull null

\nid We furthermore introduce a generic constant for a $\circnull$ reference.
%
\begin{gendef}[X]
  \circnull : \circreftype X
\end{gendef}
%
The {\OhCircus} keyword $\circclass$ is used synonymously for $\circprocess$.

\begin{verbatim}
%%Zword \circclass \circprocess
\end{verbatim}
%
For a class specification to parse, we note, however, that we have to introduce a main action which, to avoid confusion, should be hidden by a comment block.

The keywords $\circinitial$, $\circpublic$, $\circprotected$, $\circprivate$, $\circlogical$, $\circfunction$ and $\circsync$ are treated as white spaces for the time being. This is until we have a working {\OhCircus} parser and type-checker.

\begin{verbatim}
%%Zword \circinitial {}
%%Zword \circpublic {}
%%Zword \circprotected {}
%%Zword \circprivate {}
%%Zword \circlogical {}
%%Zword \circfunction {}
%%Zword \circsync {}
\end{verbatim}

\subsection{{\Circus} Time}

The absolute and relative time domains are introduced below.
%
\begin{zed}
  TIME ~ == ~ \nat
  \also
  PERIOD ~ == ~ \nat
\end{zed}

\nid
We also provide support for the supplementary {\SCJCircus} deadline operators.
%
\begin{verbatim}
%%Zpreword \circdeadlinesync \circdeadline
%%Zpreword \circdeadlineterm \circdeadline
\end{verbatim}

\nid
The following channel is used to parse $c~\circat~t \then A$ which is translated into a prefix (hack).
%
\begin{circus}
  \circchannel ~ circat : TIME
\end{circus}
%
\begin{verbatim}
%%Zword \circat {\then circat~?~}
\end{verbatim}

\subsection{Special Commands}

The commands \verb"\hidemath" and \verb"\circstateignore" are treated as white spaces.
%
\begin{verbatim}
%%Zword \hidemath {}
%%Zword \circstateignore {}
\end{verbatim}

\nid
We also ignore various commands for colouring text.
%
\begin{verbatim}
%%Zword \red {}
%%Zword \blue {}
%%Zword \green {}
%%Zword \purple {}
%%Zword \grey {}
\end{verbatim}

\nid
The commands \verb"\circblockopen" and \verb"\circblockclose" have the same meaning as normal parentheses but produce a block layout.
%
\begin{verbatim}
%%Zword \circblockopen {(}
%%Zword \circblockclose {)}
\end{verbatim}

\nid
Similarly, \verb"\biglcurly" and \verb"\bigrcurly" have the same meaning as curly brackets around set expressions but again produce a block layout.
%
\begin{verbatim}
%%Zword \biglcurly {\{}
%%Zword \bigrcurly {\}}
\end{verbatim}

\nid The `$\dots$' command is treated as $\Skip$.

\begin{verbatim}
%%Zword \dots \Skip
\end{verbatim}


\newpage

%%%%%%%%%%%%%%%%%%%%%%%%%%%%%%%%%%%%%%%%%%%%%%%%%%%%%%%%%%%%%%%%%%%%%%%%%%%%%%%%
%                                   Safelet                                    %
%%%%%%%%%%%%%%%%%%%%%%%%%%%%%%%%%%%%%%%%%%%%%%%%%%%%%%%%%%%%%%%%%%%%%%%%%%%%%%%%

\section{Safelet}

\subsection{Channels}

\begin{circusbox}
\input{SafeletChan.circus}
\end{circusbox}

\subsubsection*{Framework channels}

\begin{circusbox}
\input{SafeletFWChan.circus}
\end{circusbox}

\subsubsection*{Method channels}

\begin{circusbox}
\input{SafeletMethChan.circus}
\end{circusbox}

\subsection{Framework Process}

\begin{circusbox}
\input{SafeletFW.circus}
\end{circusbox}

\subsection{Application Process}

\begin{circusbox}
\input{MainSafeletApp.circus}
\end{circusbox}

\subsection{Composite Process}

\begin{circusbox}
\input{MainSafelet.circus}
\end{circusbox}

\newpage

%%%%%%%%%%%%%%%%%%%%%%%%%%%%%%%%%%%%%%%%%%%%%%%%%%%%%%%%%%%%%%%%%%%%%%%%%%%%%%%%
%                              Mission Sequencer                               %
%%%%%%%%%%%%%%%%%%%%%%%%%%%%%%%%%%%%%%%%%%%%%%%%%%%%%%%%%%%%%%%%%%%%%%%%%%%%%%%%

\section{Mission Sequencer}

\subsection{Channels}

\begin{circusbox}
\input{MissionSequencerChan.circus}
\end{circusbox}

\subsubsection*{Framework channels}

\begin{circusbox}
\input{MissionSequencerFWChan.circus}
\end{circusbox}

\subsubsection*{Method channels}

\begin{circusbox}
\input{MissionSequencerMethChan.circus}
\end{circusbox}

\subsection{Framework Process}

\begin{circusbox}
\input{MissionSequencerFW.circus}
\end{circusbox}

\subsection{Application Process}

\begin{circusbox}
\input{MainMissionSequencerApp.circus}
\end{circusbox}

\subsection{Composite Process}

\begin{circusbox}
\input{MainMissionSequencer.circus}
\end{circusbox}

\newpage

%%%%%%%%%%%%%%%%%%%%%%%%%%%%%%%%%%%%%%%%%%%%%%%%%%%%%%%%%%%%%%%%%%%%%%%%%%%%%%%%
%                                   Mission                                    %
%%%%%%%%%%%%%%%%%%%%%%%%%%%%%%%%%%%%%%%%%%%%%%%%%%%%%%%%%%%%%%%%%%%%%%%%%%%%%%%%

\section{Mission}

The life-cycle of a mission consists of first executing its \code{initialize()} method. This creates the event handlers and other data objects that are shared by the mission's handlers. The mission then enters its execution phase in which the event handlers become active, and are released either periodically or in response to external or software events being fired that are associated with the handlers. The dispatch loop continues until one of the handlers requests termination by calling \code{requestTermination()} on the current \code{Mission} object. The handlers are then stopped and the mission subsequently enters a cleanup phase and terminates.

\subsection{Channels}

\input{MissionChan.circus}

\subsubsection*{Framework channels}
\reducevspaceaftersection

\input{MissionFWChan.circus}

\subsubsection*{Method channels}
\reducevspaceaftersection

\input{MissionMethChan.circus}

\newpage

\subsection{Framework Process}

The framework process for mission execution is given below.
%
\begin{circusbox}
\input{MissionFW.circus}
\end{circusbox}

\newpage

\subsection{Application Process}

The application process for the \code{MainMission} class of the cruise controller is given below.
%
\begin{circusflow}
\input{MainMissionApp.circus}
\end{circusflow}

%%%%%%%%%%%%%%%%%%%%%%%
% REVIEWED UNTIL HERE %
%%%%%%%%%%%%%%%%%%%%%%%

%\todo{Verify the synchronisation channel set below. Not entirely sure about it.}

\subsection{Composite Process}

The composite model for the \code{MainMission} class of the cruise controller is given below.
%
\begin{circusbox}
\input{MainMission.circus}
\end{circusbox}

\newpage

%%%%%%%%%%%%%%%%%%%%%%%%%%%%%%%%%%%%%%%%%%%%%%%%%%%%%%%%%%%%%%%%%%%%%%%%%%%%%%%%
%                                 Event Handler                                %
%%%%%%%%%%%%%%%%%%%%%%%%%%%%%%%%%%%%%%%%%%%%%%%%%%%%%%%%%%%%%%%%%%%%%%%%%%%%%%%%

\section{Event Handlers}

In this section we present the framework model for both aperiodic and periodic event handlers.

\subsection{Channels}

We separately discuss the framework and method channels for handlers.

\subsubsection*{Framework channels}
\reducevspaceaftersection

\input{HandlerFWChan.circus}

\subsubsection*{Method channels}
\reducevspaceaftersection

\input{HandlerMethChan.circus}

\subsubsection*{Handler channels}

The channel sets below are introduced to facilitate the definition of composite processes for handlers.
%
\begin{circusbox}
\input{HandlerChan.circus}
\end{circusbox}

\subsection{Framework Process}

\begin{circusbox}
\input{EventHandlerFW.circus}
\end{circusbox}

\newpage

%%%%%%%%%%%%%%%%%%%%%%%%%%%%%%%%%%%%%%%%%%%%%%%%%%%%%%%%%%%%%%%%%%%%%%%%%%%%%%%%
%                            Aperiodic Event Handler                           %
%%%%%%%%%%%%%%%%%%%%%%%%%%%%%%%%%%%%%%%%%%%%%%%%%%%%%%%%%%%%%%%%%%%%%%%%%%%%%%%%

\section{Aperiodic Event Handlers}

In this section we present the framework and application models for aperiodic event handlers of the cruise controller application.

%\subsection{Framework Process}

%\begin{circusbox}
%\input{AperiodicEventHandlerFW.circus}
%\end{circusbox}

\subsection{Framework Processes}

In this section we illustrates the instantiation of the framework process for the aperiodic event handlers of the cruise controller in order to obtain the models for particular aperiodic handlers.

\subsubsection{WheelShaft}

\begin{circusbox}
\input{WheelShaftFW.circus}
\end{circusbox}

\subsubsection{Engine}

\begin{circusbox}
\input{EngineFW.circus}
\end{circusbox}

\subsubsection{Brake}

\begin{circusbox}
\input{BrakeFW.circus}
\end{circusbox}

\subsubsection{Gear}

\begin{circusbox}
\input{GearFW.circus}
\end{circusbox}

\subsubsection{Lever}

\begin{circusbox}
\input{LeverFW.circus}
\end{circusbox}

\newpage

\subsection{Application Processes}

In this section we illustrate the application model for aperiodic event handlers by defining a process for each aperiodic handler of the cruise controller. They all have very similar shapes.

\subsubsection{WheelShaft}

\begin{circusbox}
\input{WheelShaftApp.circus}
\end{circusbox}

\paragraph{Note}

\red{The channels for methods calls to \code{void handleAsyncEvent()} and \code{void handleAsyncLongEvent(int value)} seem redundant with the most recent revision of the model. Furthermore, I have not encoded the $handlerAsyncLongEvent$ method as an action here but data operations. What policy we adopt here is still subject to discussion. Clearly, however, since there no outputs a data operations is sufficient here.}

\subsubsection{Engine}

\begin{circusbox}
\input{EngineApp.circus}
\end{circusbox}

\subsubsection{Brake}

\begin{circusbox}
\input{BrakeApp.circus}
\end{circusbox}

\subsubsection{Gear}

\begin{circusbox}
\input{GearApp.circus}
\end{circusbox}

\subsubsection{Lever}

\begin{circusbox}
\input{LeverApp.circus}
\end{circusbox}

\newpage

\subsection{Composite Processes}

\subsubsection{WheelShaft}

\begin{circusbox}
\input{WheelShaft.circus}
\end{circusbox}

\subsubsection{Engine}

\begin{circusbox}
\input{Engine.circus}
\end{circusbox}

\subsubsection{Brake}

\begin{circusbox}
\input{Brake.circus}
\end{circusbox}

\subsubsection{Gear}

\begin{circusbox}
\input{Gear.circus}
\end{circusbox}

\subsubsection{Lever}

\begin{circusbox}
\input{Lever.circus}
\end{circusbox}

\newpage

%%%%%%%%%%%%%%%%%%%%%%%%%%%%%%%%%%%%%%%%%%%%%%%%%%%%%%%%%%%%%%%%%%%%%%%%%%%%%%%%
%                            Periodic Event Handler                            %
%%%%%%%%%%%%%%%%%%%%%%%%%%%%%%%%%%%%%%%%%%%%%%%%%%%%%%%%%%%%%%%%%%%%%%%%%%%%%%%%

\section{Periodic Event Handlers}

In this section we present the framework and application model for periodic event handlers.

%\subsection{Framework Process}

%\begin{circusbox}
%\input{PeriodicEventHandlerFW.circus}
%\end{circusbox}

\subsection{Framework Processes}

\subsubsection{SpeedMonitor}

\begin{circusbox}
\input{SpeedMonitorFW.circus}
\end{circusbox}

\subsubsection{ThrottleController}

\begin{circusbox}
\input{ThrottleControllerFW.circus}
\end{circusbox}

\subsection{Application Processes}

The application processes for handlers basically all have the same shape. They result from a lifting of the underlying data object. The latter is modelled by an {\OhCircus} class.

\subsubsection{SpeedMonitor}

\begin{circusbox}
\input{SpeedMonitorApp.circus}
\end{circusbox}

\subsubsection{ThrottleController}

\begin{circusbox}
\input{ThrottleControllerApp.circus}
\end{circusbox}

\subsubsection{ThrottleController -- Alternative 1}

\begin{circusbox}
\input{ThrottleControllerApp1.circus}
\end{circusbox}

\subsubsection{ThrottleController -- Alternative 2}

\begin{circusbox}
\input{ThrottleControllerApp2.circus}
\end{circusbox}

\subsection{Composite Processes}

\subsubsection{SpeedMonitor}

\begin{circusbox}
\input{SpeedMonitor.circus}
\end{circusbox}

\subsubsection{ThrottleController}

\begin{circusbox}
\input{ThrottleController.circus}
\end{circusbox}

\newpage

%%%%%%%%%%%%%%%%%%%%%%%%%%%%%%%%%%%%%%%%%%%%%%%%%%%%%%%%%%%%%%%%%%%%%%%%%%%%%%%%
%                                  SCJ Events                                  %
%%%%%%%%%%%%%%%%%%%%%%%%%%%%%%%%%%%%%%%%%%%%%%%%%%%%%%%%%%%%%%%%%%%%%%%%%%%%%%%%

\section{SCJ Events}

SCJ events model \emph{software events}. In the program they are instances of the \code{AperiodicEvent} class.

\subsection{Constants}

\begin{circusbox}
\input{SCJEventId.circus}
\end{circusbox}
%
\begin{circusbox}
\input{SCJEventConst.circus}
\end{circusbox}

\subsection{Channels}

\begin{circusbox}
\input{SCJEventFWChan.circus}
\end{circusbox}
%
\begin{circusbox}
\input{SCJEventMethChan.circus}
\end{circusbox}

\subsection{Framework Processes}

\begin{circusbox}
\input{SCJEventFW.circus}
\end{circusbox}

\newpage

%%%%%%%%%%%%%%%%%%%%%%%%%%%%%%%%%%%%%%%%%%%%%%%%%%%%%%%%%%%%%%%%%%%%%%%%%%%%%%%%
%                                  Framework                                   %
%%%%%%%%%%%%%%%%%%%%%%%%%%%%%%%%%%%%%%%%%%%%%%%%%%%%%%%%%%%%%%%%%%%%%%%%%%%%%%%%

%\section{Framework}
\vspace{-0.75em}

\begin{zsection}
  \SECTION ~ scj\_framework ~ \parents\\
  \t1 SafeletFW,\\
  \t1 MissionSequencerFW,\\
  \t1 MissionFW,\\
  \t1 AperiodicEventHandlerFW,\\
  \t1 PeriodicEventHandlerFW
\end{zsection}


%%%%%%%%%%%%%%%%%%%%%%%%%%%%%%%%%%%%%%%%%%%%%%%%%%%%%%%%%%%%%%%%%%%%%%%%%%%%%%%%
%                                   Toolkit                                    %
%%%%%%%%%%%%%%%%%%%%%%%%%%%%%%%%%%%%%%%%%%%%%%%%%%%%%%%%%%%%%%%%%%%%%%%%%%%%%%%%

%\section{Toolkit}
\vspace{-0.75em}

\begin{zsection}
  \SECTION ~ scj\_toolkit ~ \parents ~ scj\_prelude, scj\_framework
\end{zsection}


%%%%%%%%%%%%%%%%%%%%%%%%%%%%%%%%%%%%%%%%%%%%%%%%%%%%%%%%%%%%%%%%%%%%%%%%%%%%%%%%
%                              Cruise Controller                               %
%%%%%%%%%%%%%%%%%%%%%%%%%%%%%%%%%%%%%%%%%%%%%%%%%%%%%%%%%%%%%%%%%%%%%%%%%%%%%%%%

\section{Cruise Controller}

This section contains definitions of constants and channels specific to the cruise controller example.

\subsection{Constants}
\reducevspaceaftersection

\input{Constants.circus}

\subsection{Events}

Below we declare a channel for each sensor and actuator event of the cruise controller.
%
%\begin{circusbox}
\input{Events.circus}
%\end{circusbox}

\subsection{Mission Identifiers}

Mission identifiers for the cruise controller application (we only have one mission).
%
\begin{circusbox}
\input{MissionIds.circus}
\end{circusbox}

\subsection{Handler Identifiers}

Handler identifiers for the cruise controller application.
%
\begin{circusbox}
\input{HandlerIds.circus}
\end{circusbox}

\pagebreak

\subsection{Event Identifiers}

Event identifiers for the cruise controller application.
%
\begin{circusbox}
\input{EventIds.circus}
\end{circusbox}

\newpage

\subsection{Top-level Model}

In this section we specify the top-level model of the entire cruise controller application. It is the composition of the safelet, mission sequencer, mission, as well as all handler components.
%
%\begin{circusflow}
\documentclass{article}
\usepackage{fullpage}
\usepackage{wasysym}
\usepackage{verbatim}
\usepackage[usenames,dvipsnames]{color}
\usepackage[colour]{circus}
\usepackage{hijac}

\title{{\Circus} Model for the Cruise Controller}

\author{Frank Zeyda}

\begin{document}

\maketitle

\tableofcontents

\newpage

\section{Constants and Events}

This section includes definitions of constants and channels for the automotive cruise controller example.

\subsection{Constants}

\input{Constants.circus}

\subsection{Events}

Below we declare a channel for each sensor and actuator event of the cruise controller.
%
\input{Events.circus}

\subsection{Mission Identifiers}

Mission identifiers for the cruise controller application. (We only have one mission.)

\begin{circusbox}
\input{MissionIds.circus}
\end{circusbox}

\subsection{Handler Identifiers}

Handler identifiers for the cruise controller application.

\begin{circusbox}
\input{HandlerIds.circus}
\end{circusbox}

\subsection{Event Identifiers}

Event identifiers for the cruise controller application.
%
\begin{circusbox}
\input{EventIds.circus}
\end{circusbox}

\newpage

%%%%%%%%%%%%%%%%%%%%%%%%%%%%%%%%%%%%%%%%%%%%%%%%%%%%%%%%%%%%%%%%%%%%%%%%%%%%%%%%
%                            Aperiodic Event Handler                           %
%%%%%%%%%%%%%%%%%%%%%%%%%%%%%%%%%%%%%%%%%%%%%%%%%%%%%%%%%%%%%%%%%%%%%%%%%%%%%%%%

\section{Aperiodic Event Handlers}

In this section we present the framework and application models for aperiodic event handlers of the cruise controller application.

%\subsection{Framework Process}

%\begin{circusbox}
%\input{AperiodicEventHandlerFW.circus}
%\end{circusbox}

\subsection{Framework Processes}

In this section we illustrates the instantiation of the framework process for the aperiodic event handlers of the cruise controller in order to obtain the models for particular aperiodic handlers.

\subsubsection{WheelShaft}

\begin{circusbox}
\input{WheelShaftFW.circus}
\end{circusbox}

\subsubsection{Engine}

\begin{circusbox}
\input{EngineFW.circus}
\end{circusbox}

\subsubsection{Brake}

\begin{circusbox}
\input{BrakeFW.circus}
\end{circusbox}

\subsubsection{Gear}

\begin{circusbox}
\input{GearFW.circus}
\end{circusbox}

\subsubsection{Lever}

\begin{circusbox}
\input{LeverFW.circus}
\end{circusbox}

\newpage

\subsection{Application Processes}

In this section we illustrate the application model for aperiodic event handlers by defining a process for each aperiodic handler of the cruise controller. They all have very similar shapes.

\subsubsection{WheelShaft}

\newpage

\begin{circusbox}
\input{../../circus/WheelShaftApp.circus}
\end{circusbox}

\newpage

\begin{circusbox}
\input{WheelShaftApp.circus}
\end{circusbox}

\newpage

%\paragraph{Note}

%\red{The channels for methods channels for \code{void handleAsyncEvent()} and \code{void handleAsyncLongEvent(int value)} seem redundant with the most recent revision of the model. Furthermore, I have not encoded the $handlerAsyncLongEvent$ method as an action here but data operations. What policy we adopt here is still subject to discussion. Clearly, however, since there no outputs a data operations is sufficient here.}

\subsubsection*{Interrupt Handler}

\verbatiminput{accs/interrupts/WheelShaftInterruptHandler.java}

\newpage

\subsubsection{Engine}

\begin{circusbox}
\input{EngineApp.circus}
\end{circusbox}

\subsubsection*{Interrupt Handler}

\verbatiminput{accs/interrupts/EngineInterruptHandler.java}

\nid \red{A problem here is that the types do not agree: whereas the channel type is $BOOL$ the value that is passed to the handler is of type $long$. There hence has to be a conversion from channel types to $long$.}

\newpage

\subsubsection{Brake}

\begin{circusbox}
\input{BrakeApp.circus}
\end{circusbox}

\subsubsection*{Interrupt Handler}

\verbatiminput{accs/interrupts/BrakeInterruptHandler.java}

\nid \red{Here we again assume an implicit mapping from events to identifiers. To unify this case with the previous one for $Engine$, we could make the event id an implicit input. So although the underlying channels are not parametrised, a parameter is nevertheless passed to the handler.}

\newpage

\subsubsection{Gear}

\begin{circusbox}
\input{GearApp.circus}
\end{circusbox}

\nid \green{Exactly the same case as $Brake$ --- the handler parameter is implicit.}

\subsubsection*{Interrupt Handler}

\verbatiminput{accs/interrupts/GearInterruptHandler.java}

\newpage

\subsubsection{Lever}

\begin{circusbox}
\input{LeverApp.circus}
\end{circusbox}

\subsubsection*{Interrupt Handler}

\verbatiminput{accs/interrupts/LeverInterruptHandler.java}

\newpage

\subsection{Composite Processes}

\subsubsection{WheelShaft}

\begin{circusbox}
\input{WheelShaft.circus}
\end{circusbox}

\subsubsection{Engine}

\begin{circusbox}
\input{Engine.circus}
\end{circusbox}

\subsubsection{Brake}

\begin{circusbox}
\input{Brake.circus}
\end{circusbox}

\subsubsection{Gear}

\begin{circusbox}
\input{Gear.circus}
\end{circusbox}

\subsubsection{Lever}

\begin{circusbox}
\input{Lever.circus}
\end{circusbox}

\newpage

%%%%%%%%%%%%%%%%%%%%%%%%%%%%%%%%%%%%%%%%%%%%%%%%%%%%%%%%%%%%%%%%%%%%%%%%%%%%%%%%
%                            Periodic Event Handler                            %
%%%%%%%%%%%%%%%%%%%%%%%%%%%%%%%%%%%%%%%%%%%%%%%%%%%%%%%%%%%%%%%%%%%%%%%%%%%%%%%%

\section{Periodic Event Handlers}

In this section we present the framework and application model for periodic event handlers.

%\subsection{Framework Process}

%\begin{circusbox}
%\input{PeriodicEventHandlerFW.circus}
%\end{circusbox}

\subsection{Framework Processes}

\subsubsection{SpeedMonitor}

\begin{circusbox}
\input{SpeedMonitorFW.circus}
\end{circusbox}

\subsubsection{ThrottleController}

\begin{circusbox}
\input{ThrottleControllerFW.circus}
\end{circusbox}

\subsection{Application Processes}

The application processes for handlers basically all have the same shape. They result from a lifting of the underlying data object. The latter is modelled by an {\OhCircus} class.

\subsubsection{SpeedMonitor}

\begin{circusbox}
\input{SpeedMonitorApp.circus}
\end{circusbox}

\subsubsection{ThrottleController}

\begin{circusbox}
\input{ThrottleControllerApp.circus}
\end{circusbox}

\newpage

\subsubsection{ThrottleController -- Alternative 1}

\begin{circusbox}
\input{ThrottleControllerApp1.circus}
\end{circusbox}

\newpage

\subsubsection{ThrottleController -- Alternative 2}

\begin{circusbox}
\input{ThrottleControllerApp2.circus}
\end{circusbox}

\newpage

\subsection{Composite Processes}

\subsubsection{SpeedMonitor}

\begin{circusbox}
\input{SpeedMonitor.circus}
\end{circusbox}

\subsubsection{ThrottleController}

\begin{circusbox}
\input{ThrottleController.circus}
\end{circusbox}

\newpage

\subsection{Top-level Model}

In this section we specify the top-level model of the entire cruise controller application. It is the composition of the safelet, mission sequencer, mission, as well as all handler components.

\begin{circusbox}
\documentclass{article}
\usepackage{fullpage}
\usepackage{wasysym}
\usepackage{verbatim}
\usepackage[usenames,dvipsnames]{color}
\usepackage[colour]{circus}
\usepackage{hijac}

\title{{\Circus} Model for the Cruise Controller}

\author{Frank Zeyda}

\begin{document}

\maketitle

\tableofcontents

\newpage

\section{Constants and Events}

This section includes definitions of constants and channels for the automotive cruise controller example.

\subsection{Constants}

\input{Constants.circus}

\subsection{Events}

Below we declare a channel for each sensor and actuator event of the cruise controller.
%
\input{Events.circus}

\subsection{Mission Identifiers}

Mission identifiers for the cruise controller application. (We only have one mission.)

\begin{circusbox}
\input{MissionIds.circus}
\end{circusbox}

\subsection{Handler Identifiers}

Handler identifiers for the cruise controller application.

\begin{circusbox}
\input{HandlerIds.circus}
\end{circusbox}

\subsection{Event Identifiers}

Event identifiers for the cruise controller application.
%
\begin{circusbox}
\input{EventIds.circus}
\end{circusbox}

\newpage

%%%%%%%%%%%%%%%%%%%%%%%%%%%%%%%%%%%%%%%%%%%%%%%%%%%%%%%%%%%%%%%%%%%%%%%%%%%%%%%%
%                            Aperiodic Event Handler                           %
%%%%%%%%%%%%%%%%%%%%%%%%%%%%%%%%%%%%%%%%%%%%%%%%%%%%%%%%%%%%%%%%%%%%%%%%%%%%%%%%

\section{Aperiodic Event Handlers}

In this section we present the framework and application models for aperiodic event handlers of the cruise controller application.

%\subsection{Framework Process}

%\begin{circusbox}
%\input{AperiodicEventHandlerFW.circus}
%\end{circusbox}

\subsection{Framework Processes}

In this section we illustrates the instantiation of the framework process for the aperiodic event handlers of the cruise controller in order to obtain the models for particular aperiodic handlers.

\subsubsection{WheelShaft}

\begin{circusbox}
\input{WheelShaftFW.circus}
\end{circusbox}

\subsubsection{Engine}

\begin{circusbox}
\input{EngineFW.circus}
\end{circusbox}

\subsubsection{Brake}

\begin{circusbox}
\input{BrakeFW.circus}
\end{circusbox}

\subsubsection{Gear}

\begin{circusbox}
\input{GearFW.circus}
\end{circusbox}

\subsubsection{Lever}

\begin{circusbox}
\input{LeverFW.circus}
\end{circusbox}

\newpage

\subsection{Application Processes}

In this section we illustrate the application model for aperiodic event handlers by defining a process for each aperiodic handler of the cruise controller. They all have very similar shapes.

\subsubsection{WheelShaft}

\newpage

\begin{circusbox}
\input{../../circus/WheelShaftApp.circus}
\end{circusbox}

\newpage

\begin{circusbox}
\input{WheelShaftApp.circus}
\end{circusbox}

\newpage

%\paragraph{Note}

%\red{The channels for methods channels for \code{void handleAsyncEvent()} and \code{void handleAsyncLongEvent(int value)} seem redundant with the most recent revision of the model. Furthermore, I have not encoded the $handlerAsyncLongEvent$ method as an action here but data operations. What policy we adopt here is still subject to discussion. Clearly, however, since there no outputs a data operations is sufficient here.}

\subsubsection*{Interrupt Handler}

\verbatiminput{accs/interrupts/WheelShaftInterruptHandler.java}

\newpage

\subsubsection{Engine}

\begin{circusbox}
\input{EngineApp.circus}
\end{circusbox}

\subsubsection*{Interrupt Handler}

\verbatiminput{accs/interrupts/EngineInterruptHandler.java}

\nid \red{A problem here is that the types do not agree: whereas the channel type is $BOOL$ the value that is passed to the handler is of type $long$. There hence has to be a conversion from channel types to $long$.}

\newpage

\subsubsection{Brake}

\begin{circusbox}
\input{BrakeApp.circus}
\end{circusbox}

\subsubsection*{Interrupt Handler}

\verbatiminput{accs/interrupts/BrakeInterruptHandler.java}

\nid \red{Here we again assume an implicit mapping from events to identifiers. To unify this case with the previous one for $Engine$, we could make the event id an implicit input. So although the underlying channels are not parametrised, a parameter is nevertheless passed to the handler.}

\newpage

\subsubsection{Gear}

\begin{circusbox}
\input{GearApp.circus}
\end{circusbox}

\nid \green{Exactly the same case as $Brake$ --- the handler parameter is implicit.}

\subsubsection*{Interrupt Handler}

\verbatiminput{accs/interrupts/GearInterruptHandler.java}

\newpage

\subsubsection{Lever}

\begin{circusbox}
\input{LeverApp.circus}
\end{circusbox}

\subsubsection*{Interrupt Handler}

\verbatiminput{accs/interrupts/LeverInterruptHandler.java}

\newpage

\subsection{Composite Processes}

\subsubsection{WheelShaft}

\begin{circusbox}
\input{WheelShaft.circus}
\end{circusbox}

\subsubsection{Engine}

\begin{circusbox}
\input{Engine.circus}
\end{circusbox}

\subsubsection{Brake}

\begin{circusbox}
\input{Brake.circus}
\end{circusbox}

\subsubsection{Gear}

\begin{circusbox}
\input{Gear.circus}
\end{circusbox}

\subsubsection{Lever}

\begin{circusbox}
\input{Lever.circus}
\end{circusbox}

\newpage

%%%%%%%%%%%%%%%%%%%%%%%%%%%%%%%%%%%%%%%%%%%%%%%%%%%%%%%%%%%%%%%%%%%%%%%%%%%%%%%%
%                            Periodic Event Handler                            %
%%%%%%%%%%%%%%%%%%%%%%%%%%%%%%%%%%%%%%%%%%%%%%%%%%%%%%%%%%%%%%%%%%%%%%%%%%%%%%%%

\section{Periodic Event Handlers}

In this section we present the framework and application model for periodic event handlers.

%\subsection{Framework Process}

%\begin{circusbox}
%\input{PeriodicEventHandlerFW.circus}
%\end{circusbox}

\subsection{Framework Processes}

\subsubsection{SpeedMonitor}

\begin{circusbox}
\input{SpeedMonitorFW.circus}
\end{circusbox}

\subsubsection{ThrottleController}

\begin{circusbox}
\input{ThrottleControllerFW.circus}
\end{circusbox}

\subsection{Application Processes}

The application processes for handlers basically all have the same shape. They result from a lifting of the underlying data object. The latter is modelled by an {\OhCircus} class.

\subsubsection{SpeedMonitor}

\begin{circusbox}
\input{SpeedMonitorApp.circus}
\end{circusbox}

\subsubsection{ThrottleController}

\begin{circusbox}
\input{ThrottleControllerApp.circus}
\end{circusbox}

\newpage

\subsubsection{ThrottleController -- Alternative 1}

\begin{circusbox}
\input{ThrottleControllerApp1.circus}
\end{circusbox}

\newpage

\subsubsection{ThrottleController -- Alternative 2}

\begin{circusbox}
\input{ThrottleControllerApp2.circus}
\end{circusbox}

\newpage

\subsection{Composite Processes}

\subsubsection{SpeedMonitor}

\begin{circusbox}
\input{SpeedMonitor.circus}
\end{circusbox}

\subsubsection{ThrottleController}

\begin{circusbox}
\input{ThrottleController.circus}
\end{circusbox}

\newpage

\subsection{Top-level Model}

In this section we specify the top-level model of the entire cruise controller application. It is the composition of the safelet, mission sequencer, mission, as well as all handler components.

\begin{circusbox}
\documentclass{article}
\usepackage{fullpage}
\usepackage{wasysym}
\usepackage{verbatim}
\usepackage[usenames,dvipsnames]{color}
\usepackage[colour]{circus}
\usepackage{hijac}

\title{{\Circus} Model for the Cruise Controller}

\author{Frank Zeyda}

\begin{document}

\maketitle

\tableofcontents

\newpage

\section{Constants and Events}

This section includes definitions of constants and channels for the automotive cruise controller example.

\subsection{Constants}

\input{Constants.circus}

\subsection{Events}

Below we declare a channel for each sensor and actuator event of the cruise controller.
%
\input{Events.circus}

\subsection{Mission Identifiers}

Mission identifiers for the cruise controller application. (We only have one mission.)

\begin{circusbox}
\input{MissionIds.circus}
\end{circusbox}

\subsection{Handler Identifiers}

Handler identifiers for the cruise controller application.

\begin{circusbox}
\input{HandlerIds.circus}
\end{circusbox}

\subsection{Event Identifiers}

Event identifiers for the cruise controller application.
%
\begin{circusbox}
\input{EventIds.circus}
\end{circusbox}

\newpage

%%%%%%%%%%%%%%%%%%%%%%%%%%%%%%%%%%%%%%%%%%%%%%%%%%%%%%%%%%%%%%%%%%%%%%%%%%%%%%%%
%                            Aperiodic Event Handler                           %
%%%%%%%%%%%%%%%%%%%%%%%%%%%%%%%%%%%%%%%%%%%%%%%%%%%%%%%%%%%%%%%%%%%%%%%%%%%%%%%%

\section{Aperiodic Event Handlers}

In this section we present the framework and application models for aperiodic event handlers of the cruise controller application.

%\subsection{Framework Process}

%\begin{circusbox}
%\input{AperiodicEventHandlerFW.circus}
%\end{circusbox}

\subsection{Framework Processes}

In this section we illustrates the instantiation of the framework process for the aperiodic event handlers of the cruise controller in order to obtain the models for particular aperiodic handlers.

\subsubsection{WheelShaft}

\begin{circusbox}
\input{WheelShaftFW.circus}
\end{circusbox}

\subsubsection{Engine}

\begin{circusbox}
\input{EngineFW.circus}
\end{circusbox}

\subsubsection{Brake}

\begin{circusbox}
\input{BrakeFW.circus}
\end{circusbox}

\subsubsection{Gear}

\begin{circusbox}
\input{GearFW.circus}
\end{circusbox}

\subsubsection{Lever}

\begin{circusbox}
\input{LeverFW.circus}
\end{circusbox}

\newpage

\subsection{Application Processes}

In this section we illustrate the application model for aperiodic event handlers by defining a process for each aperiodic handler of the cruise controller. They all have very similar shapes.

\subsubsection{WheelShaft}

\newpage

\begin{circusbox}
\input{../../circus/WheelShaftApp.circus}
\end{circusbox}

\newpage

\begin{circusbox}
\input{WheelShaftApp.circus}
\end{circusbox}

\newpage

%\paragraph{Note}

%\red{The channels for methods channels for \code{void handleAsyncEvent()} and \code{void handleAsyncLongEvent(int value)} seem redundant with the most recent revision of the model. Furthermore, I have not encoded the $handlerAsyncLongEvent$ method as an action here but data operations. What policy we adopt here is still subject to discussion. Clearly, however, since there no outputs a data operations is sufficient here.}

\subsubsection*{Interrupt Handler}

\verbatiminput{accs/interrupts/WheelShaftInterruptHandler.java}

\newpage

\subsubsection{Engine}

\begin{circusbox}
\input{EngineApp.circus}
\end{circusbox}

\subsubsection*{Interrupt Handler}

\verbatiminput{accs/interrupts/EngineInterruptHandler.java}

\nid \red{A problem here is that the types do not agree: whereas the channel type is $BOOL$ the value that is passed to the handler is of type $long$. There hence has to be a conversion from channel types to $long$.}

\newpage

\subsubsection{Brake}

\begin{circusbox}
\input{BrakeApp.circus}
\end{circusbox}

\subsubsection*{Interrupt Handler}

\verbatiminput{accs/interrupts/BrakeInterruptHandler.java}

\nid \red{Here we again assume an implicit mapping from events to identifiers. To unify this case with the previous one for $Engine$, we could make the event id an implicit input. So although the underlying channels are not parametrised, a parameter is nevertheless passed to the handler.}

\newpage

\subsubsection{Gear}

\begin{circusbox}
\input{GearApp.circus}
\end{circusbox}

\nid \green{Exactly the same case as $Brake$ --- the handler parameter is implicit.}

\subsubsection*{Interrupt Handler}

\verbatiminput{accs/interrupts/GearInterruptHandler.java}

\newpage

\subsubsection{Lever}

\begin{circusbox}
\input{LeverApp.circus}
\end{circusbox}

\subsubsection*{Interrupt Handler}

\verbatiminput{accs/interrupts/LeverInterruptHandler.java}

\newpage

\subsection{Composite Processes}

\subsubsection{WheelShaft}

\begin{circusbox}
\input{WheelShaft.circus}
\end{circusbox}

\subsubsection{Engine}

\begin{circusbox}
\input{Engine.circus}
\end{circusbox}

\subsubsection{Brake}

\begin{circusbox}
\input{Brake.circus}
\end{circusbox}

\subsubsection{Gear}

\begin{circusbox}
\input{Gear.circus}
\end{circusbox}

\subsubsection{Lever}

\begin{circusbox}
\input{Lever.circus}
\end{circusbox}

\newpage

%%%%%%%%%%%%%%%%%%%%%%%%%%%%%%%%%%%%%%%%%%%%%%%%%%%%%%%%%%%%%%%%%%%%%%%%%%%%%%%%
%                            Periodic Event Handler                            %
%%%%%%%%%%%%%%%%%%%%%%%%%%%%%%%%%%%%%%%%%%%%%%%%%%%%%%%%%%%%%%%%%%%%%%%%%%%%%%%%

\section{Periodic Event Handlers}

In this section we present the framework and application model for periodic event handlers.

%\subsection{Framework Process}

%\begin{circusbox}
%\input{PeriodicEventHandlerFW.circus}
%\end{circusbox}

\subsection{Framework Processes}

\subsubsection{SpeedMonitor}

\begin{circusbox}
\input{SpeedMonitorFW.circus}
\end{circusbox}

\subsubsection{ThrottleController}

\begin{circusbox}
\input{ThrottleControllerFW.circus}
\end{circusbox}

\subsection{Application Processes}

The application processes for handlers basically all have the same shape. They result from a lifting of the underlying data object. The latter is modelled by an {\OhCircus} class.

\subsubsection{SpeedMonitor}

\begin{circusbox}
\input{SpeedMonitorApp.circus}
\end{circusbox}

\subsubsection{ThrottleController}

\begin{circusbox}
\input{ThrottleControllerApp.circus}
\end{circusbox}

\newpage

\subsubsection{ThrottleController -- Alternative 1}

\begin{circusbox}
\input{ThrottleControllerApp1.circus}
\end{circusbox}

\newpage

\subsubsection{ThrottleController -- Alternative 2}

\begin{circusbox}
\input{ThrottleControllerApp2.circus}
\end{circusbox}

\newpage

\subsection{Composite Processes}

\subsubsection{SpeedMonitor}

\begin{circusbox}
\input{SpeedMonitor.circus}
\end{circusbox}

\subsubsection{ThrottleController}

\begin{circusbox}
\input{ThrottleController.circus}
\end{circusbox}

\newpage

\subsection{Top-level Model}

In this section we specify the top-level model of the entire cruise controller application. It is the composition of the safelet, mission sequencer, mission, as well as all handler components.

\begin{circusbox}
\input{accs.circus}
\end{circusbox}

\newpage

%%%%%%%%%%%%%%%%%%%%%%%%%%%%%%%%%%%%%%%%%%%%%%%%%%%%%%%%%%%%%%%%%%%%%%%%%%%%%%%%
%                                 Data Objects                                 %
%%%%%%%%%%%%%%%%%%%%%%%%%%%%%%%%%%%%%%%%%%%%%%%%%%%%%%%%%%%%%%%%%%%%%%%%%%%%%%%%

\section{Data Objects}

In this section we give the {\OhCircus} class definitions for all data objects of the Cruise Controller case study. It also illustrates how the respective Java classes are translated into formal models.

Not all of the specification can be parsed at the moment due to limitations of CZT to understand {\OhCircus}. The parts that are not subjected to the parser are highlighted in dark red.

\subsection{WheelShaft}

\input{WheelShaftClass.circus}

\newpage

\subsection{Engine}

\input{EngineClass.circus}

\newpage

\subsection{Brake}

\input{BrakeClass.circus}

\newpage

\subsection{Gear}

\input{GearClass.circus}

\newpage

\subsection{Lever}

\input{LeverClass.circus}

\newpage

\subsection{SpeedMonitor}

\input{SpeedMonitorClass.circus}

\newpage

\subsection{ThrottleController}

\input{ThrottleControllerClass.circus}

\newpage

\subsection{CruiseControl}

\input{CruiseControlClass.circus}

\nid The methods called by this class are:
%
\begin{itemize}
  \item \code{ThrottleController: void setCruiseSpeed(int speed)}
  \item \code{ThrottleController: void accelerate()}
  \item \code{ThrottleController: void schedulePeriodic()}
  \item \code{ThrottleController: void deschedulePeriodic()}
  \item \code{SpeedMonitor: int getCurrentSpeed()}
\end{itemize}

\end{document}

\end{circusbox}

\newpage

%%%%%%%%%%%%%%%%%%%%%%%%%%%%%%%%%%%%%%%%%%%%%%%%%%%%%%%%%%%%%%%%%%%%%%%%%%%%%%%%
%                                 Data Objects                                 %
%%%%%%%%%%%%%%%%%%%%%%%%%%%%%%%%%%%%%%%%%%%%%%%%%%%%%%%%%%%%%%%%%%%%%%%%%%%%%%%%

\section{Data Objects}

In this section we give the {\OhCircus} class definitions for all data objects of the Cruise Controller case study. It also illustrates how the respective Java classes are translated into formal models.

Not all of the specification can be parsed at the moment due to limitations of CZT to understand {\OhCircus}. The parts that are not subjected to the parser are highlighted in dark red.

\subsection{WheelShaft}

\input{WheelShaftClass.circus}

\newpage

\subsection{Engine}

\input{EngineClass.circus}

\newpage

\subsection{Brake}

\input{BrakeClass.circus}

\newpage

\subsection{Gear}

\input{GearClass.circus}

\newpage

\subsection{Lever}

\input{LeverClass.circus}

\newpage

\subsection{SpeedMonitor}

\input{SpeedMonitorClass.circus}

\newpage

\subsection{ThrottleController}

\input{ThrottleControllerClass.circus}

\newpage

\subsection{CruiseControl}

\input{CruiseControlClass.circus}

\nid The methods called by this class are:
%
\begin{itemize}
  \item \code{ThrottleController: void setCruiseSpeed(int speed)}
  \item \code{ThrottleController: void accelerate()}
  \item \code{ThrottleController: void schedulePeriodic()}
  \item \code{ThrottleController: void deschedulePeriodic()}
  \item \code{SpeedMonitor: int getCurrentSpeed()}
\end{itemize}

\end{document}

\end{circusbox}

\newpage

%%%%%%%%%%%%%%%%%%%%%%%%%%%%%%%%%%%%%%%%%%%%%%%%%%%%%%%%%%%%%%%%%%%%%%%%%%%%%%%%
%                                 Data Objects                                 %
%%%%%%%%%%%%%%%%%%%%%%%%%%%%%%%%%%%%%%%%%%%%%%%%%%%%%%%%%%%%%%%%%%%%%%%%%%%%%%%%

\section{Data Objects}

In this section we give the {\OhCircus} class definitions for all data objects of the Cruise Controller case study. It also illustrates how the respective Java classes are translated into formal models.

Not all of the specification can be parsed at the moment due to limitations of CZT to understand {\OhCircus}. The parts that are not subjected to the parser are highlighted in dark red.

\subsection{WheelShaft}

\input{WheelShaftClass.circus}

\newpage

\subsection{Engine}

\input{EngineClass.circus}

\newpage

\subsection{Brake}

\input{BrakeClass.circus}

\newpage

\subsection{Gear}

\input{GearClass.circus}

\newpage

\subsection{Lever}

\input{LeverClass.circus}

\newpage

\subsection{SpeedMonitor}

\input{SpeedMonitorClass.circus}

\newpage

\subsection{ThrottleController}

\input{ThrottleControllerClass.circus}

\newpage

\subsection{CruiseControl}

\input{CruiseControlClass.circus}

\nid The methods called by this class are:
%
\begin{itemize}
  \item \code{ThrottleController: void setCruiseSpeed(int speed)}
  \item \code{ThrottleController: void accelerate()}
  \item \code{ThrottleController: void schedulePeriodic()}
  \item \code{ThrottleController: void deschedulePeriodic()}
  \item \code{SpeedMonitor: int getCurrentSpeed()}
\end{itemize}

\end{document}

%\end{circusflow}

\newpage

%%%%%%%%%%%%%%%%%%%%%%%%%%%%%%%%%%%%%%%%%%%%%%%%%%%%%%%%%%%%%%%%%%%%%%%%%%%%%%%%
%                                 Data Objects                                 %
%%%%%%%%%%%%%%%%%%%%%%%%%%%%%%%%%%%%%%%%%%%%%%%%%%%%%%%%%%%%%%%%%%%%%%%%%%%%%%%%

\section{Data Objects}

In this section we give the {\OhCircus} class definitions for all data objects of the Cruise Controller case study. It also illustrates how the respective Java classes are translated into formal models.

Not all of the specification can be parsed at the moment due to limitations of CZT to understand {\OhCircus}. The parts that are not subjected to the parser are highlighted in dark red.

\subsection{WheelShaft}
\reducevspaceaftersection

\input{WheelShaftClass.circus}

\newpage

\subsection{Engine}
\reducevspaceaftersection

\input{EngineClass.circus}

\newpage

\subsection{Brake}
\reducevspaceaftersection

\input{BrakeClass.circus}

\newpage

\subsection{Gear}
\reducevspaceaftersection

\input{GearClass.circus}

\newpage

\subsection{Lever}
\reducevspaceaftersection

\input{LeverClass.circus}

\newpage

\subsection{SpeedMonitor}
\reducevspaceaftersection

\input{SpeedMonitorClass.circus}

\newpage

\subsection{ThrottleController}
\reducevspaceaftersection

\input{ThrottleControllerClass.circus}

\newpage

\subsection{CruiseControl}
\reducevspaceaftersection

\input{CruiseControlClass.circus}

\newpage

\nid The methods called by this class are:
%
\begin{itemize}
  \item \code{ThrottleController: void setCruiseSpeed(int speed)}
  \item \code{ThrottleController: void accelerate()}
  \item \code{ThrottleController: void schedulePeriodic()}
  \item \code{ThrottleController: void deschedulePeriodic()}
  \item \code{SpeedMonitor: int getCurrentSpeed()}
\end{itemize}
\end{document}
