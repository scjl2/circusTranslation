\documentclass[11pt,a4paper]{article}
\usepackage[utf8]{inputenc}
\usepackage[english]{babel}
\usepackage{amsmath}
\usepackage{amsfonts}
\usepackage{amssymb}
\usepackage{graphicx}

\usepackage{xcolor}

\usepackage{circus}
\usepackage{hijac}
\usepackage{czt}


\usepackage{listings}
\lstset{ %
mathescape,
language=Java,                % choose the language of the code
basicstyle=\footnotesize,       % the size of the fonts that are used for the code
numbers=left,                   % where to put the line-numbers
numberstyle=\footnotesize,      % the size of the fonts that are used for the line-numbers
stepnumber=1,                   % the step between two line-numbers. If it is 1 each line will be numbered
numbersep=5pt,                  % how far the line-numbers are from the code
showspaces=false,               % show spaces adding particular underscores
showstringspaces=false,         % underline spaces within strings
showtabs=false,                 % show tabs within strings adding particular underscores
frame=none,           % adds a frame around the code
tabsize=2,          % sets default tabsize to 2 spaces
captionpos=b,           % sets the caption-position to bottom
breaklines=true,        % sets automatic line breaking
breakatwhitespace=false,    % sets if automatic breaks should only happen at whitespace
escapeinside={\%*}{*)}          % if you want to add a comment within your code
}

\usepackage[left=40mm,right=15mm,top=15mm,bottom=15mm]{geometry}
\author{Matt Luckcuck}
\begin{document}

\section*{Translation Rules}

\subsection*{High Level}

\input{HighLevelRules.circus}
 \newpage
\subsection*{Low Level}
\begin{itemize}
\item $Method : MethodDeclaration \pfun (Name, Params, ReturnType, Body)$ : translates an active application method into a \Circus{} action
\item $DataMethod : MethodDeclaration \pfun  $ : translates data methods into an \textbf{\textit{Oh}}\Circus{} method
\item $MethodBody : Block \pfun \seq CircExpression $ : translates a Java block, for example a method body
\item $Registers : Block \pfun \seq Name $ : extracts the Names of the schedulables registered in a Java block
\item $Returns : Block \pfun \seq Name $ : extracts the Names of the variables retuned in a Java block
\item $Variable : (Name, Type, InitExpression) \pfun (CircName, CircType, CircExpression) $ : translates a variable
\item $Parameters : (Name, Params, ReturnType, Body) \pfun \seq CircParam $ : translates a list of method parameters
\end{itemize}

\begin{itemize}
\item $\lpar Name \rpar_{name}$ : translates the $name$ to a Z identifier
\item $\lpar varType \rpar_{type}$ : translates types
\item $\lpar expr \rpar_{expression}$ : translates expressions
\end{itemize}

\newpage
\section*{Auxiliary Functions}

\begin{itemize}
\item $IdOf(name)$: yields the identifier of a component called $name$
\item $ObjectIdOf(name)$: yields the identifier of the $Object$ process of a component called $name$
\item $ThreadIdOf(name)$: yields the identifier of the $Thread$ process of a component called $name$
\item $MethodName(method)$: yields the method name of $method$
\end{itemize}
\newpage
\section*{Pattern Matching Rules}

\section*{Safelet}
\begin{lstlisting}
public class $Identifier$ implements Safelet
{
	$FieldDeclaration\_1$
	...
	$FieldDeclaration\_n$

	$ConstructorDeclaration$

	$initializeApplication$

	$getSequencer$

	$AppMeth\_1$
	...
	$AppMeth\_n$
}
\end{lstlisting}

\begin{circus}
\circprocess \lpar Identifier \rpar_{Name}App  \circdef \lpar \lpar ConstructorDeclaration \rpar_{Method}\rpar_{Parameters} \circbegin
\end{circus}

%
\begin{schema}{State}
 	this : \circref \lpar PName \rpar_{name}
\end{schema}
% 
\begin{circusaction}
\circstate State
\end{circusaction}
%
\begin{schema}{Init}
  State~' \\
  this := \circnew \lpar PName \rpar_{name} Class()
\end{schema}
%

\begin{circusaction}
InitializeApplication \circdef \\
\circblockopen
     initializeApplicationCall \then \\

	  \lpar \lpar InitializeApplication \rpar_{Method} \rpar_{MethBody} \\

     initializeApplicationRet \then \\
     \Skip
\circblockclose
\end{circusaction}

\begin{circusaction}
GetSequencer \circdef \\
\circblockopen
	getSequencerCall \then \\
	getSequencerRet~!~ \lpar GetSequencer \rpar_{Returns} \then \\
	\Skip
\circblockclose
\end{circusaction}

\begin{circusaction}
\lpar AppMeth\_1 \rpar_{Method}
\end{circusaction}
\qquad \ldots
\begin{circusaction}
\lpar AppMeth\_n \rpar_{Method}
\end{circusaction}

\begin{circusaction}
Methods \circdef \\
\circblockopen
	GetSequencer \\
	\extchoice  \\
	InitializeApplication  \\
	\extchoice \\
	MethName(AppMeth\_1) \\
	\extchoice \\
  \ldots \\
	MethName(AppMeth\_n) \\
	\ldots
\circblockclose
\circseq Methods
\end{circusaction}

$\circspot (\lschexpract Init \rschexpract \circseq Methods) $
 $\circinterrupt (end\_safelet\_app \then \Skip)$

\begin{circus}
  \circend
\end{circus}


\newpage

\section*{Mission Sequencer}

\begin{lstlisting}
public class $Identifier$ extends MissionSequencer
{
	$FieldDeclaration\_1$
	...
	$FieldDeclaration\_n$

	$ConstructorDeclaration$

	$getNextMission$

	$AppMeth\_1$
	...
	$AppMeth\_n$
}
\end{lstlisting}
\begin{circus}
\circprocess \lpar PName \rpar_{name} App \circdef \lpar PParams \rpar_{params} \circbegin\\
\end{circus}

%
\begin{schema}{State}
 	this : \circref \lpar PName \rpar_{name}
\end{schema}
% 
\begin{circusaction}
\circstate State
\end{circusaction}
%
\begin{schema}{Init}
  State~' \\
  this := \circnew \lpar PName \rpar_{name} Class()
\end{schema}
%

\begin{circusaction}
GetNextMission \circdef \circvar ret : MissionID \circspot \\
\circblockopen
    getNextMissionCall~.~IdOf(PName) \then \\
	   ret := this~.~getNextMission() \circseq \\
    getNextMissionRet~.~IdOf(PName)~!~ret  \then \\
\Skip
\circblockclose
\end{circusaction}

$\lpar AppMeth1 \rpar_{appMeth}$

$\lpar AppMeth2 \rpar_{appMeth}$

\ldots

\begin{circusaction}
Methods \circdef  \\
\circblockopen
	GetNextMission \\
	\extchoice \\
	MethName(AppMeth1) \\
	\extchoice \\
	MethName(AppMeth2) \\
	\ldots 
\circblockclose
\circseq Methods
\end{circusaction}


$\circspot (\lschexpract Init \rschexpract \circseq Methods) $
  $\circinterrupt (end\_sequencer\_app~.~IdOf(PName)  \then \Skip)$


\begin{circus}
  \circend
\end{circus}


%TODO MSAPP Class

\newpage

\section*{Mission}

\begin{lstlisting}
public class $Identifier$ extends Mission
{
	$FieldDeclaration\_1$
	...
	$FieldDeclaration\_n$

	$ConstructorDeclaration$

	$initialize$

	$cleanUp$

	$AppMeth\_1$
	...
	$AppMeth\_n$
}
\end{lstlisting}
\begin{circus}
\circprocess \lpar PName \rpar App \circdef \lpar PParams \rpar_{params} \circbegin
\end{circus}

%
\begin{schema}{State}
 	this : \circref \lpar PName \rpar_{name}
\end{schema}
% 
\begin{circusaction}
\circstate State
\end{circusaction}
%
\begin{schema}{Init}
  State~' \\
  this := \circnew \lpar PName \rpar_{name} Class()
\end{schema}
%

\begin{circusaction}
InitializePhase \circdef \\
\circblockopen
  initializeCall~.~IdOf(PName)  \then \\

	\lpar RegisteredSchedulables \rpar


	%register~!~${SchedulableID}~!~${ProcessID} \then   \\



  initializeRet~.~IdOf(PName)  \then \\
  \Skip
\circblockclose
\end{circusaction}

\begin{circusaction}
CleanupPhase \circdef  \\
\circblockopen
 cleanupMissionCall~.~IdOf(PName)  \then \\

 cleanupMissionRet~.~IdOf(PName) ~!~\true \then \\
 \Skip
\circblockclose
\end{circusaction}

$\lpar AppMeth1 \rpar_{appMeth}$

$\lpar AppMeth2 \rpar_{appMeth}$

\ldots

\begin{circusaction}
Methods \circdef
\circblockopen
	InitializePhase \\
	\extchoice \\
	CleanupPhase \\
	\extchoice \\
	MethName(AppMeth1) \\
	\extchoice \\
	MethName(AppMeth2) \\
	\ldots
\circblockclose
\circseq Methods
\end{circusaction}


$\circspot (\lschexpract Init \rschexpract \circseq Methods) $
 $\circinterrupt (end\_mission\_app~.~IdOf(PName) \then \Skip$ 

\begin{circus}
  \circend
\end{circus}

\newpage

\section*{Handlers}

\begin{lstlisting}
class $Identifier$ extends $HandlerType$
{
	$FieldDeclaration\_1$
	...
	$FieldDeclaration\_n$

	$ConstructorDeclaration$

	$handleAsyncEvent$

	$AppMeth\_1$
	...
	$AppMeth\_n$
}
\end{lstlisting}
\begin{circus}
\circprocess \lpar PName \rpar App \circdef \lpar PParams \rpar_{params}  \circbegin
\end{circus}

%
\begin{schema}{State}
 	this : \circref \lpar PName \rpar_{name}
\end{schema}
% 
\begin{circusaction}
\circstate State
\end{circusaction}
%
\begin{schema}{Init}
  State~' \\
  this := \circnew \lpar PName \rpar_{name} Class()
\end{schema}
%

\begin{circusaction}
handleAsyncEvent \circdef \\
\circblockopen
	handleAsyncEventCall~.~IdOf(PName) \then \\
	\lpar HandleAsyncBody \rpar \circseq \\
    handleAsyncEventRet~.~IdOf(PName) \then \\
    \Skip
\circblockclose
\end{circusaction}

$\lpar AppMeth1 \rpar_{appMeth}$

$\lpar AppMeth2 \rpar_{appMeth}$

\ldots

\begin{circusaction}
Methods \circdef \\
\circblockopen
	handleAsyncEvent \\
MethName(AppMeth1) \\
	\extchoice \\
	MethName(AppMeth2) \\
	\ldots
\circblockclose
	 \circseq Methods
\end{circusaction}


$\circspot (\lschexpract Init \rschexpract \circseq Methods) $
 $ \circinterrupt (end\_ \lpar HandlerType IdOf(PName) \rpar \then \Skip)$


\begin{circus}
  \circend
\end{circus}

\newpage

\section*{Managed Thread}

\begin{lstlisting}
public class $Identifier$ extends ManagedThread
{
	$FieldDeclaration\_1$
	...
	$FieldDeclaration\_n$

	$ConstructorDeclaration$

	$run$

	$AppMeth\_1$
	...
	$AppMeth\_n$
}
\end{lstlisting}
\begin{circus}
\circprocess \lpar PName \rpar App \circdef \lpar PParams \rpar_{params} \circbegin
\end{circus}

%
\begin{schema}{State}
 	this : \circref \lpar PName \rpar_{name}
\end{schema}
% 
\begin{circusaction}
\circstate State
\end{circusaction}
%
\begin{schema}{Init}
  State~' \\
  this := \circnew \lpar PName \rpar_{name} Class()
\end{schema}
%

\begin{circusaction}
Run \circdef \\
\circblockopen
	runCall~.~IdOf(PName) \then \\
	\lpar RunBody \rpar \circseq \\
	runRet~.~IfOf(PName) \then \\
	\Skip
\circblockclose
\end{circusaction}

$\lpar AppMeth1 \rpar_{appMeth}$

$\lpar AppMeth2 \rpar_{appMeth}$

\ldots

\begin{circusaction}
Methods \circdef \\
\circblockopen
	Run \\
\extchoice \\
	MethName(AppMeth1) \\
	\extchoice \\
	MethName(AppMeth2) \\
	\ldots
\circblockclose
	 \circseq Methods
\end{circusaction}

$\circspot (\lschexpract Init \rschexpract \circseq Methods) $
 $ \circinterrupt (end\_managedThread\_app~.~IdOf(PName) \then \Skip)$

\begin{circus}
  \circend
\end{circus}


\section*{Data Class}

\begin{circus}
\circclass \lpar PName \rpar_{name} Class \circdef \circbegin
\end{circus}


\begin{schema}{\circstateignore State}

  \lpar VarName \rpar_{name} : \lpar VarType \rpar_{type}

\end{schema}
% 
\begin{circusaction}
\circstate State
\end{circusaction}
%
\begin{schema}{\circinitial Init}
  State~'
\where

  \lpar VarName \rpar_{name} ' = \lpar VarInit \rpar_{expression} 

\end{schema}



$\lpar DataMeth1 \rpar_{dataMeth}$

$\lpar DataMeth2 \rpar_{dataMeth}$

\ldots

\begin{circusaction}
\circspot ~ \Skip
\end{circusaction}

\begin{circus}
  \circend
\end{circus}


\end{document}
